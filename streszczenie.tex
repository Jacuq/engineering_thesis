\pdfbookmark[0]{Streszczenie}{streszczenie.1}
%\phantomsection
%\addcontentsline{toc}{chapter}{Streszczenie}
%%% Poni�sze zosta�o niewykorzystane (tj. zrezygnowano z utworzenia nienumerowanego rozdzia�u na abstrakt)
%%%\begingroup
%%%\setlength\beforechapskip{48pt} % z jakiego� powodu by�a male�ka r�nica w po�o�eniu nag��wka rozdzia�u numerowanego i nienumerowanego
%%%\chapter*{\centering Abstrakt}
%%%\endgroup
%%%\label{sec:abstrakt}
%%%Lorem ipsum dolor sit amet eleifend et, congue arcu. Morbi tellus sit amet, massa. Vivamus est id risus. Sed sit amet, libero. Aenean ac ipsum. Mauris vel lectus. 
%%%
%%%Nam id nulla a adipiscing tortor, dictum ut, lobortis urna. Donec non dui. Cras tempus orci ipsum, molestie quis, lacinia varius nunc, rhoncus purus, consectetuer congue risus. 
%\mbox{}\vspace{2cm} % mo�na przesun��, w zale�no�ci od d�ugo�ci streszczenia
\begin{abstract}
Ninejsza praca in�ynierska to dokumnetcja projektowa aplikacji webowej opartej na REST API s�u��cej do wspierania muzycznego samokszta�cenia. Aplikacja ma pomaga� muzykom w opanowaniu teorii muzyki przy pomocy wirtualnych instrument�w oraz interaktywnych �wicze�. Praca skupia si� przede wszystkim na stronie technicznej projektu. Pierwsze rozdzia�y s�u�� om�wieniu zastosowanych technologii, architektury i podj�tych decyzji projektowych. Nast�pnie om�wiona jest implementacja systemu. Na ko�cu przedstawione s� zrealizowane cele oraz perspektywy na rozw�j aplikacji. 


\end{abstract}
\mykeywords{aplikacja internetowa, Angular, TypeScript, Python, Flask, REST API, teoria muzyki}\\ 
% Dobrze by�oby skopiowa� s�owa kluczowe do metadanych dokumentu pdf (w pliku Dyplom.tex)
% Niestety, zaimplementowane makro nie robi tego z automatu, wi�c pozostaje kopiowanie r�czne.

{
\selectlanguage{english}
\begin{abstract}
This bachelor thesis is a design document of a web application based on REST API. Application is designed to help musicians in developing their knowledge of music theory by using virtual instruments and interactive excercises. The document focuses on technical aspect of application rather than music theory. First chapters are focused on describing decisionmaking process used while making decisions about system architecture, choosing technologies and project design. Next chapters are used to present implementation details and final results. The document ends by describing future prospects for the app. 

Nam id nulla a adipiscing tortor, dictum ut, lobortis urna. Donec non dui. Cras tempus orci ipsum, molestie quis, lacinia varius nunc, rhoncus purus, consectetuer congue risus. 
\end{abstract}
\mykeywords{web application, Angular, TypeScript, Flask, Python, REST API, music theory}
}
